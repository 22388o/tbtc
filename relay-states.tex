\documentclass{article}
\usepackage[utf8]{inputenc}

\usepackage[margin=1in]{geometry}
\usepackage{tikz}
\usepackage[colorlinks=true]{hyperref}
\usepackage{varwidth}

\usetikzlibrary{positioning}
\usetikzlibrary{arrows.meta}
\usetikzlibrary{shapes.symbols}
\usetikzlibrary{calc}

\tikzset{
  every node/.style={above},
  start state/.style={draw,circle,text width=0},
  state/.style={draw,circle,align=flush center,text width=2cm},
  decision/.style={draw,rectangle,align=flush center},
  thread/.style={draw,signal,signal to=east,fill=white},
  % Nested states and decisions use doubled borders to indicate
  % that you have to reference a different diagram.
  nested state/.style={draw,circle,double,align=flush center},
  nested decision/.style={draw,rectangle,double,align=flush center},
  % Chain states and decisions use dashed borders to indicate
  % that you have to interact with the chain.
  chain state/.style={draw,circle,dashed,align=flush center,text width=2cm},
  chain decision/.style={draw,rectangle,dashed,align=flush center},
  chain transition/.style={draw,dashed},
  % Nested chain decisions combine nested and chain styles to
  % indicate a reference to a different diagram for a state that
  % depends on chain interaction.
  nested chain decision/.style={draw,rectangle,double,dashed,align=flush center},
  % Set default arrow style to stealth.
  >=Stealth
}

\begin{document}

Below are state diagrams elaborating on the various pieces of a Keep node. Note
that Keep nodes actually run multiple processes in parallel, for participation
in the random beacon threshold relay as well as for participating in the Keep
network. The startup diagram indicates this and links to the subprocess state
diagrams.

First, however, a legend:

\vspace{0.5cm}

\begin{center}
  \tikz{
      \node[thread] (thread) {A thread};

      \node[nested state] (nested state) [below=1cm of thread] {A nested state};
      \node[state] [left=1cm of nested state](state) {A regular state};
      \node[chain state] (chain state) [right=1cm of nested state] {A state dependent on the chain};

      \node[nested decision] (nested decision) [below=1cm of nested state] {A nested decision};
      \node[decision] (decision) [left=1cm of nested decision] {A decision};
      \node[chain decision,text width=2cm] (chain decision) [right=1cm of nested decision] {A decision dependent on the chain};

      \draw [<-] (-4.5,0.25) -- node [centered,text width=2cm,align=center] {state\\transition} (-2.5,0.25);
      \draw [->,dashed] (2,0.25) -- node [centered,text width=4cm,align=center] {chain-dependent\\state transition} (5.25,0.25)
  }
\end{center}

\vspace{0.5cm}

Nested states and decisions should link to the nested diagrams that indicate
what is going on within these states or decisions.

\vspace{0.5cm}

\begin{center}
  % -*- root: relay-states-preview.tex -*-

\tikz[every node/.style={node distance=2cm},
      fan out/.style={to path={-- ++(0.5,0) -| ($ (\tikztotarget) - (1,0) $) -- (\tikztotarget)}}]{
  \node[draw,circle] (start) {};

  \node[thread] (group participation) [above right=2cm and 2cm of start] {\hyperref[fig:relay-group-initialization]{Relay group participation}};
  \node[thread] (verification) [right=of start] {\hyperref[fig:relay-entry-verification]{Relay entry verification}};
  \node[thread] (keep client) [below right=2cm and 2cm of start] {\hyperref[fig:keep-client-initialization]{Keep client}};

  \path [->] (start) edge [fan out]  (group participation.west)
             (start) edge [fan out]  (verification.west)
             (start) edge [fan out]  (keep client.west);
}


  \vspace{0.5cm}
  Keep node startup process: processes run in parallel
\end{center}



\clearpage

\begin{figure}
  \centering
  % !TEX root = ../relay-states.tex
\tikz{
  \node[start state] (start) {};

  \node[chain decision] (stake check) [right=of start] {Staked?\footnote{Local check to avoid doing unnecessary work.}};

  \node[nested state] (joining) [right=of stake check,text width=2cm] {\hyperref[fig:libp2p-join]{Joining\\\tt{libp2p}}};

  \node[chain decision] (state check) [below=2cm of joining] {Checking\\current state};

  \node[nested state] (waiting) [below left=3cm of state check,text width=2cm] {\hyperref[fig:relay-group-assignment]{Waiting\\for group}};
  \node[nested state] (setting up group) [below right=3cm of waiting,text width=2cm] {\hyperref[fig:relay-group-setup]{Setting up group}};

  \node[nested state] (processing) [below right=3cm of state check,text width=1.7cm] {\hyperref[fig:relay-entry-request-processing]{Processing requests}};

  \path [->] (start) edge (stake check)

             (stake check) edge [chain transition,bend right=45] node [above] {No} (start)
             (stake check) edge [chain transition] node [above] {Yes} (joining)

             % NOTE We may not want to return to the stake check on failed join,
             % NOTE we may just want to abort.
             (joining) edge [bend right=45] node [above] {failed to join} (stake check)
             (joining) edge node {joined} (state check)

             (state check) edge [chain transition,bend right=40] node [left] {no group} (waiting)
                           edge [chain transition] node [pos=0.6,text width=3cm,align=flush center] {in uninitialized group} (setting up group)
                           edge [chain transition,bend left=40] node [right] {in initialized group} (processing)

             (waiting) edge [bend left=40] node [left,pos=0.75,text width=2cm,align=flush center] {assigned to group} (setting up group)

             (setting up group) [bend left=40] edge node [left] {setup failed} (waiting)
             (setting up group) [bend right=40] edge node [right] {group activated} (processing)

             (processing) [bend left=40] edge node [right,pos=0.75,text width=2cm,align=flush center] {group dissolved} (state check)
}
  
  \caption{\label{fig:relay-group-initialization}Relay Group Initialization}
\end{figure}

\begin{figure}
  \centering
  % !TEX root = ../relay-states.tex
\tikz{
  \node [nested decision] (state check) [text width=2cm] {\hyperref[fig:relay-group-initialization]{Checking\\current state}};

  \node [state] (waiting for entry) [right=2cm of state check] {Waiting for relay entry};

  \node [decision] (checking) [right=3cm of waiting for entry] {Checking eligibility*};

  \node [nested state] (setting up group) [right=2cm of checking,text width=2cm] {\hyperref[fig:relay-group-setup]{Setting up group}};

  \path[->] (state check) edge node {no group} (waiting for entry)

            (waiting for entry) edge node [centered,text width=2cm,align=center] {relay entry published} (checking)

            (checking) edge [bend right=45] node {not eligible} (waiting for entry)
                       edge node {eligible} (setting up group);
}

  
  \caption{\label{fig:relay-group-assignment}Relay Group Assignment}

  \vspace{0.5cm}
  % Mildly abusing the itemize env...
  \begin{itemize}
  \item[*] Note that eligibility checking is done by combining the latest relay
           entry with the latest participant registry in a way specified outside
           this document.
  \end{itemize}
\end{figure}

\begin{figure}
  \centering
  % !TEX root = ../relay-states.tex
\tikz{
  \node[nested state,text width=2cm] (waiting) {\hyperref[fig:relay-group-initialization]{Waiting for group}};

  \node[chain state] (awaiting entry) [right=of waiting] {Awaiting\\relay entry};

  \node[decision] (eligibility check) [right=2.5cm of awaiting entry] {Eligible for group?};

  \node[state] (joining channel) [right=2cm of eligibility check] {Joining broadcast channel};

  \node[state] (generating) [below left=1cm and 3cm of joining channel] {Generating secret key share and proofs};

  \node[state] (announcing) [left=2cm of generating] {Announcing secret key proofs};

  \node[decision] (verifying) [below=1cm of announcing] {Verifying other proofs};

  % FIXME Accusation process is incomplete and might be its own sub-diagram.
  \node[state] (accusing) [below right=-0.8cm and 2cm of verifying] {Publishing invalidity accusation};
  \node[state] (generating pubkey) [below left=2cm of verifying.south] {Generating public key};

  \node[chain state] (submitting pubkey) [right=2cm of generating pubkey] {Submitting public key};

  \node[chain state] (awaiting pubkey) [below=1cm of submitting pubkey] {Awaiting on-chain public key};
  
  \node[state] (pending activation delay) [right=2cm of awaiting pubkey] {Waiting for activation delay};

  \node[nested state] (processing) [below=4cm of joining channel,text width=2cm] {\hyperref[fig:relay-entry-request-processing]{Processing requests}};

   \path [->] (waiting) edge (awaiting entry)

              (awaiting entry) edge [chain transition] node [above] {entry received} (eligibility check)

              (eligibility check) edge [bend right=45] node [above] {No} (awaiting entry)
                                  edge [pos=0.6] node {Yes} (joining channel)

              (joining channel) edge [bend right=30] node {joining failed} (waiting)
                                edge node [sloped] {joined} (generating)

               (generating) edge node [centered,text width=1.5cm,align=flush center] {shares generated} (announcing)
              
               (announcing) edge [pos=0.65] node {shares announced} (verifying)
              
               (verifying) edge node [centered,text width=1.5cm,align=flush center] {invalid share} (accusing)
                           edge [pos=0.65] node {shares valid} (generating pubkey)

               (generating pubkey) edge node {generated} (submitting pubkey)

               (submitting pubkey) edge [chain transition,pos=0.65] node {submitted} (awaiting pubkey)
              
               (awaiting pubkey) edge [chain transition,to path={[rounded corners=3cm] -| (\tikztotarget) \tikztonodes}] node [pos=0.15,centered,text width=2cm] {relay entry received} (waiting)
                                 edge [chain transition] node [centered,text width=1.5cm,align=flush center] {pubkey published} (pending activation delay)

               (pending activation delay) edge [bend right=15] node [right,text width=2cm] {activation delay elapsed} (processing)
}
  
  \caption{\label{fig:relay-group-setup}Relay Group Setup}
\end{figure}

\begin{figure}
  \centering
  % !TEX root = ../relay-states.tex
\tikz{
    \node[nested state] (initialization) [text width=1.7cm] {\hyperref[fig:relay-group-initialization]{Pending Activation}};

    \node[chain state] (waiting) [right=of initialization,text width=1.5cm] {Waiting for request};

    \node[decision] (determining) [right=3cm of waiting,text width=2cm] {Is group responsible?};

    \node[state] (generating) [right=2cm of determining,text width=1.7cm] {Generating signature share};

    % Brodcasting? Might just be part of generating.
    \node[state] (verifying shares) [below=1cm of generating,text width=1.7cm] {Verifying shares};

    \node[chain state] (submitting) [left=3cm of verifying shares,text width=1.7cm] {Submitting signature};

    \path[->] (initialization) edge (waiting)

              (waiting) edge [chain transition] node {request received} (determining)

              (determining) edge [bend left=45] node [below] {no} (waiting)
              (determining) edge node {yes} (generating)

              (generating) edge [bend left=45] node [right,pos=0.45,text width=2cm] {share\\generated} (verifying shares)

              (verifying shares) edge node {signature ready} (submitting)

              (submitting) edge [chain transition,to path={[rounded corners=1.5cm] -| (\tikztotarget) \tikztonodes}] node [centered,pos=0.15,text width=2cm] {signature submitted} (waiting);
}

   
  \caption{\label{fig:relay-entry-request-processing}Relay Entry Request Processing}
\end{figure}

\begin{figure}
  \centering
  % !TEX root = ../relay-states.tex
\tikz{
    \node[nested chain decision] (stake check) {\hyperref[fig:relay-group-initialization]{Staked?}};

    \node[state] (connecting) [right=2cm of stake check] {Connecting to bootstrap host};

    \node[state] (proving stake) [below=1cm of connecting] {Submitting stake proof};

    \node[state] (receiving peers) [below=1cm of proving stake] {Receiving peerlist};

    \node[nested chain decision] (state check) [left=2cm of receiving peers,text width=2cm] {\hyperref[fig:relay-group-initialization]{Checking\\current state}};

    \path [->] (stake check) edge node {Yes} (connecting)

               (connecting) edge node [pos=0.6] {connected} (proving stake)
                            edge [bend right=30] node {failed to connect} (stake check)

               (proving stake) edge node [pos=0.7] {proof accepted} (receiving peers)
                               edge [bend left=45] node [right,pos=0.6,text width=1.5cm] {proof rejected} (stake check)

               (receiving peers) edge node [centered,text width=1.4cm,align=flush center] {peers received} (state check)
                                 edge [bend left=30] node [left,pos=0.3] {failed to receive peers} (stake check)
}

  \caption{\label{fig:libp2p-join}{\tt libp2p} Join Process}
\end{figure}

\begin{figure}
  \centering
  % !TEX root = ../relay-states.tex
\tikz{
  \node [start state] (start) {};

  \node [chain state] (waiting for entry) [right=of start,text width=1.8cm] {Waiting for relay entry};

  \node [decision] (verifying) [right=4cm of waiting for entry] {Is entry valid?};

  \node [chain state] (calling chain verification) [right=2cm of verifying,text width=1.8cm] {Calling on-chain verification};

  \path[->] (start) edge (waiting for entry)

            (waiting for entry) edge [chain transition] node {relay entry published} (verifying)

            (verifying) edge node {no} (calling chain verification)
            (verifying) edge [bend left=30] node [below,pos=0.55] {yes} (waiting for entry)

            % If on-chain verification fails do we back off?
            (calling chain verification) edge [chain transition,bend right=30] node {on-chain verification complete} (waiting for entry);
}


  \caption{\label{fig:relay-entry-verification}Relay Entry Verification}
\end{figure}

\end{document}
