\documentclass{article}
\usepackage[utf8]{inputenc}

\usepackage[margin=1in]{geometry}
\usepackage{tikz}
\usepackage[colorlinks=true]{hyperref}
\usepackage{varwidth}

\usetikzlibrary{positioning}
\usetikzlibrary{arrows.meta}
\usetikzlibrary{shapes.geometric}
\usetikzlibrary{shapes.symbols}
\usetikzlibrary{calc}

\tikzset{
  every node/.style={above},
  start state/.style={draw,ellipse,text width=0},
  state/.style={draw,ellipse,align=flush center,text width=2.5cm},
  decision/.style={draw,rectangle,align=flush center},
  thread/.style={draw,signal,signal to=east,fill=white},
  % Nested states and decisions use doubled borders to indicate
  % that you have to reference a different diagram.
  nested state/.style={draw,ellipse,double,align=flush center,text width=2.5cm},
  nested decision/.style={draw,rectangle,double,align=flush center},
  % Chain states and decisions use dashed borders to indicate
  % that you have to interact with the chain.
  chain state/.style={draw,ellipse,dashed,align=flush center,text width=2.5cm},
  chain decision/.style={draw,rectangle,dashed,align=flush center},
  chain transition/.style={draw,dashed},
  % Nested chain decisions combine nested and chain styles to
  % indicate a reference to a different diagram for a state that
  % depends on chain interaction.
  nested chain decision/.style={draw,rectangle,double,dashed,align=flush center},
  % Set default arrow style to stealth.
  >=Stealth
}

\begin{document}

Below are state diagrams elaborating on the various pieces of a Keep node. Note
that Keep nodes actually run multiple processes in parallel, for participation
in the random beacon threshold relay as well as for participating in the Keep
network. The startup diagram indicates this and links to the subprocess state
diagrams.

First, however, a legend:

\vspace{0.5cm}

\begin{center}
  \tikz{
      \node[thread] (thread) {A thread};

      \node[nested state] (nested state) [below=1cm of thread] {A nested state};
      \node[state] [left=1cm of nested state](state) {A regular state};
      \node[chain state] (chain state) [right=1cm of nested state] {A state dependent on the chain};

      \node[nested decision] (nested decision) [below=1cm of nested state] {A nested decision};
      \node[decision] (decision) [left=1cm of nested decision] {A decision};
      \node[chain decision,text width=2cm] (chain decision) [right=1cm of nested decision] {A decision dependent on the chain};

      \draw [<-] (-4.5,0.25) -- node [centered,text width=2cm,align=center] {state\\transition} (-2.5,0.25);
      \draw [->,dashed] (2,0.25) -- node [centered,text width=4cm,align=center] {chain-dependent\\state transition} (5.25,0.25)
  }
\end{center}

\vspace{0.5cm}

Nested states and decisions should link to the nested diagrams that indicate
what is going on within these states or decisions.

\vspace{0.5cm}

\begin{center}
  % !TEX root = ../build.tex
% !tikzlibraries = arrows.meta,calc,positioning,chains,shapes.geometric
% !include = _common.tikz
% Underline that goes through descender.

\newcommand{\evenunderline}[1]{%
  \underline{\smash{#1}\vphantom{T}}\vphantom{#1}%
}

\tikz[start chain=bitcoin going below,
      start chain=depositor going below,
      start chain=tbtc going below,
      start chain=ethereum going below,
      node distance=2cm]{
    \node [on chain=bitcoin] (bitcoin label) {\underline{Bitcoin}};
    \node [on chain=depositor] (depositor label) [right=3cm of bitcoin label] {\evenunderline{Depositor}};
    \node [on chain=tbtc] (tbtc label) [right=2cm of depositor label] {\underline{TBTC}};
    \node [on chain=ethereum] (ethereum label) [right=3cm of tbtc label] {\underline{Ethereum}};

    \node[on chain=bitcoin] (bitcoin start) {$$\vdots$$};
    \node[decision,on chain=bitcoin,join=by <-] (bitcoin block 1) {\hspace*{1cm}};
    \node[decision,on chain=bitcoin,join=by <-] (bitcoin block 2) {\hspace*{1cm}};
    \node[decision,on chain=bitcoin,join=by <-] (bitcoin block 3) {\hspace*{1cm}};
    \node[on chain=bitcoin,join=by <-] (bitcoin end) {$$\vdots$$};

    \node[on chain=ethereum] (ethereum start) {$$\vdots$$};
    \node[decision,on chain=ethereum,join=by <-] (ethereum block 1) {\hspace*{1cm}};
    \node[decision,on chain=ethereum,join=by <-] (ethereum block 2) {\hspace*{1cm}};
    \node[decision,on chain=ethereum,join=by <-] (ethereum block 3) {\hspace*{1cm}};
    \node[decision,on chain=ethereum,join=by <-] (ethereum block 4) {\hspace*{1cm}};
    \node[decision,on chain=ethereum,join=by <-] (ethereum block 5) {\hspace*{1cm}};
    \node[on chain=ethereum,join=by <-] (ethereum end) {$$\vdots$$};

    \node[state,on chain=depositor] (deposit request) at ($(depositor label)-(0,1cm)$) {request deposit creation};
    \node[nested state,on chain=tbtc] (signing group request) at ($(tbtc label)-(0,2.5cm)$) {create signing group};
    \node[state,on chain=depositor] (deposit send)    {send deposit to signing group};
    \node[state,on chain=depositor,text width=2.8cm] (deposit proof) at ($(deposit send)+(0,1cm)$)   {prove deposit transaction block};
    \node[state,on chain=tbtc]        (deposit confirmation) at ($(signing group request)-(0,2.5cm)$) {enable TBTC mint for deposit};

    \node[state,on chain=depositor] (tbtc request) at ($(deposit proof)-(0,1cm)$)   {request TBTC};
    \node[state,on chain=tbtc]      (tbtc minting)    {mint and assign TBTC};


    \path [->] (deposit request) edge [out=10,in=135] (ethereum block 1)
               (ethereum block 1) edge [dashed] (signing group request)
               (signing group request) edge [bend left=15] node [sloped,text width=2cm,align=center,anchor=center] {public wallet address} (ethereum block 2)
               (ethereum block 2) edge [dashed] (deposit send)
               (deposit send) edge [bend right=15] (bitcoin block 2)
               (bitcoin block 2) edge [bend right=15,dashed] (deposit proof)
               (deposit proof) edge [out=10,in=140] (ethereum block 3)
               (ethereum block 3) edge [out=210,in=0,dashed] (deposit confirmation)
               (tbtc request) edge (ethereum block 4)
               (ethereum block 4) edge [bend left=15] (tbtc minting);
}

  \vspace{0.5cm}
  Deposit initiation.
\end{center}



\clearpage

\begin{figure}
  \centering
  % !TEX root = ../relay-states.tex
\tikz{
    \node[nested chain decision] (stake check) {\hyperref[fig:relay-group-initialization]{Staked?}};

    \node[state] (connecting) [right=2cm of stake check] {Connecting to bootstrap host};

    \node[state] (proving stake) [below=1cm of connecting] {Submitting stake proof};

    \node[state] (receiving peers) [below=1cm of proving stake] {Receiving peerlist};

    \node[nested chain decision] (state check) [left=2cm of receiving peers,text width=2cm] {\hyperref[fig:relay-group-initialization]{Checking\\current state}};

    \path [->] (stake check) edge node {Yes} (connecting)

               (connecting) edge node [pos=0.6] {connected} (proving stake)
                            edge [bend right=30] node {failed to connect} (stake check)

               (proving stake) edge node [pos=0.7] {proof accepted} (receiving peers)
                               edge [bend left=45] node [right,pos=0.6,text width=1.5cm] {proof rejected} (stake check)

               (receiving peers) edge node [centered,text width=1.4cm,align=flush center] {peers received} (state check)
                                 edge [bend left=30] node [left,pos=0.3] {failed to receive peers} (stake check)
}

  \caption{\label{fig:libp2p-join}{\tt libp2p} Join Process}
\end{figure}

\end{document}
