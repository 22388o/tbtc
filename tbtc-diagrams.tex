\documentclass{article}
\usepackage[utf8]{inputenc}

\usepackage[margin=1in]{geometry}
\usepackage{tikz}
\usepackage[colorlinks=true]{hyperref}
\usepackage{varwidth}
\usepackage{rotating}
\usepackage{tikz-uml}
\usepackage{xstring} % for tikz-uml
\usepackage{pgfopts} % for tikz-uml
\usepackage[english]{babel} % for tikz-uml

\usetikzlibrary{positioning}
\usetikzlibrary{arrows.meta}
\usetikzlibrary{shapes.geometric}
\usetikzlibrary{shapes.symbols}
\usetikzlibrary{calc}
\usetikzlibrary{chains}
\usetikzlibrary{fit}

\tikzset{
  every node/.style={above},
  start state/.style={draw,ellipse,text width=0},
  state/.style={draw,ellipse,align=flush center,text width=2.5cm},
  box state/.style={draw,rectangle,align=flush center,text width=2.5cm,inner sep=7pt},
  leaf state/.style={box state,rounded corners=12pt},
  decision/.style={draw,rectangle,align=flush center},
  thread/.style={draw,signal,signal to=east,fill=white},
  % Nested states and decisions use doubled borders to indicate
  % that you have to reference a different diagram.
  nested state/.style={draw,ellipse,double,align=flush center,text width=2.5cm},
  nested decision/.style={draw,rectangle,double,align=flush center},
  % Chain states and decisions use dashed borders to indicate
  % that you have to interact with the chain.
  chain state/.style={draw,ellipse,dashed,align=flush center,text width=2.5cm},
  chain decision/.style={draw,rectangle,dashed,align=flush center},
  chain transition/.style={draw,dashed},
  % Nested chain decisions combine nested and chain styles to
  % indicate a reference to a different diagram for a state that
  % depends on chain interaction.
  nested chain decision/.style={draw,rectangle,double,dashed,align=flush center},
  % Set default arrow style to stealth.
  >=Stealth
}

\begin{document}

Below are state diagrams elaborating on the various pieces of a Keep node. Note
that Keep nodes actually run multiple processes in parallel, for participation
in the random beacon threshold relay as well as for participating in the Keep
network. The startup diagram indicates this and links to the subprocess state
diagrams.

First, however, a legend:

\vspace{0.5cm}

\begin{center}
  \tikz{
      \node[thread] (thread) {A thread};

      \node[nested state] (nested state) [below=1cm of thread] {A nested state};
      \node[state] [left=1cm of nested state](state) {A regular state};
      \node[chain state] (chain state) [right=1cm of nested state] {A state dependent on the chain};

      \node[nested decision] (nested decision) [below=1cm of nested state] {A nested decision};
      \node[decision] (decision) [left=1cm of nested decision] {A decision};
      \node[chain decision,text width=2cm] (chain decision) [right=1cm of nested decision] {A decision dependent on the chain};

      \draw [<-] (-4.5,0.25) -- node [centered,text width=2cm,align=center] {state\\transition} (-2.5,0.25);
      \draw [->,dashed] (2,0.25) -- node [centered,text width=4cm,align=center] {chain-dependent\\state transition} (5.25,0.25)
  }
\end{center}

\vspace{0.5cm}

Nested states and decisions should link to the nested diagrams that indicate
what is going on within these states or decisions.

\vspace{0.5cm}

\begin{figure}
  \centering
  % !TEX root = ../build.tex
% !tikzlibraries = arrows.meta,calc,positioning,chains,shapes.geometric
% !include = _common.tikz
% Underline that goes through descender.

\newcommand{\evenunderline}[1]{%
  \underline{\smash{#1}\vphantom{T}}\vphantom{#1}%
}

\tikz[start chain=bitcoin going below,
      start chain=depositor going below,
      start chain=tbtc going below,
      start chain=ethereum going below,
      node distance=2cm]{
    \node [on chain=bitcoin] (bitcoin label) {\underline{Bitcoin}};
    \node [on chain=depositor] (depositor label) [right=3cm of bitcoin label] {\evenunderline{Depositor}};
    \node [on chain=tbtc] (tbtc label) [right=2cm of depositor label] {\underline{TBTC}};
    \node [on chain=ethereum] (ethereum label) [right=3cm of tbtc label] {\underline{Ethereum}};

    \node[on chain=bitcoin] (bitcoin start) {$$\vdots$$};
    \node[decision,on chain=bitcoin,join=by <-] (bitcoin block 1) {\hspace*{1cm}};
    \node[decision,on chain=bitcoin,join=by <-] (bitcoin block 2) {\hspace*{1cm}};
    \node[decision,on chain=bitcoin,join=by <-] (bitcoin block 3) {\hspace*{1cm}};
    \node[on chain=bitcoin,join=by <-] (bitcoin end) {$$\vdots$$};

    \node[on chain=ethereum] (ethereum start) {$$\vdots$$};
    \node[decision,on chain=ethereum,join=by <-] (ethereum block 1) {\hspace*{1cm}};
    \node[decision,on chain=ethereum,join=by <-] (ethereum block 2) {\hspace*{1cm}};
    \node[decision,on chain=ethereum,join=by <-] (ethereum block 3) {\hspace*{1cm}};
    \node[decision,on chain=ethereum,join=by <-] (ethereum block 4) {\hspace*{1cm}};
    \node[decision,on chain=ethereum,join=by <-] (ethereum block 5) {\hspace*{1cm}};
    \node[on chain=ethereum,join=by <-] (ethereum end) {$$\vdots$$};

    \node[state,on chain=depositor] (deposit request) at ($(depositor label)-(0,1cm)$) {request deposit creation};
    \node[nested state,on chain=tbtc] (signing group request) at ($(tbtc label)-(0,2.5cm)$) {create signing group};
    \node[state,on chain=depositor] (deposit send)    {send deposit to signing group};
    \node[state,on chain=depositor,text width=2.8cm] (deposit proof) at ($(deposit send)+(0,1cm)$)   {prove deposit transaction block};
    \node[state,on chain=tbtc]        (deposit confirmation) at ($(signing group request)-(0,2.5cm)$) {enable TBTC mint for deposit};

    \node[state,on chain=depositor] (tbtc request) at ($(deposit proof)-(0,1cm)$)   {request TBTC};
    \node[state,on chain=tbtc]      (tbtc minting)    {mint and assign TBTC};


    \path [->] (deposit request) edge [out=10,in=135] (ethereum block 1)
               (ethereum block 1) edge [dashed] (signing group request)
               (signing group request) edge [bend left=15] node [sloped,text width=2cm,align=center,anchor=center] {public wallet address} (ethereum block 2)
               (ethereum block 2) edge [dashed] (deposit send)
               (deposit send) edge [bend right=15] (bitcoin block 2)
               (bitcoin block 2) edge [bend right=15,dashed] (deposit proof)
               (deposit proof) edge [out=10,in=140] (ethereum block 3)
               (ethereum block 3) edge [out=210,in=0,dashed] (deposit confirmation)
               (tbtc request) edge (ethereum block 4)
               (ethereum block 4) edge [bend left=15] (tbtc minting);
}

  \caption{\label{fig:deposit-initiation}Deposit initiation.}
\end{figure}

% Flush floats before a rotated page.
\clearpage

\begin{sidewaysfigure}
  \global\pdfpageattr\expandafter{\the\pdfpageattr/Rotate 90}
  \centering
  % !TEX root = ../tbtc-diagrams.tex
% Underline that goes through descender.

\begin{tikzpicture}
  \begin{umlseqdiag}
      \umlbasicobject[stereo=actor]{User}
      \umlbasicobject{Deposit Contract}
      \umlbasicobject{tBTC Keep}
      \umlbasicobject{tECDSA Keep}
      \umlbasicobject{Random Beacon Keep}
      \umlbasicobject{Ethereum Chain}

      \begin{umlcall}[op=request deposit address, dt=5]{User}{Deposit Contract}
          \begin{umlcall}[op=request deposit group, dt=0]{Deposit Contract}{tBTC Keep}
              \begin{umlcall}[type=return, op=id]{tBTC Keep}{Deposit Contract}
              \end{umlcall}
        
              \begin{umlcall}[type=return, op=id, dt=0]{Deposit Contract}{User}
              \end{umlcall}
         
              \begin{umlcall}[op=request random seed, return=seed, dt=0, padding=3]{tBTC Keep}{Random Beacon Keep}
              \end{umlcall}

              \begin{umlcallself}[op=create signing group]{tBTC Keep}
              \end{umlcallself}

              \begin{umlcall}[op=generate group keys, return=public key, padding=3]{tBTC Keep}{tECDSA Keep}
              \end{umlcall}

              \begin{umlcall}[op=broadcast group public key, type=asynchron, dt=5]{tBTC Keep}{Ethereum Chain}
              \end{umlcall}
          \end{umlcall}
      \end{umlcall}

      \begin{umlcall}[op=get deposit address, return=address, dt=5]{User}{Deposit Contract}
          \begin{umlcall}[op=get deposit group address, return=address]{Deposit Contract}{tBTC Keep}
              \begin{umlcallself}[op=convert public key to address]{tBTC Keep}
              \end{umlcallself}
          \end{umlcall}
      \end{umlcall}
  \end{umlseqdiag}
\end{tikzpicture}


  \caption{\label{fig:signing-group-creation}Signing group creation.}
\end{sidewaysfigure}

% Flush floats before a rotated page.
\clearpage

\begin{sidewaysfigure}
  \global\pdfpageattr\expandafter{\the\pdfpageattr/Rotate 90}
  \centering
  % !TEX root = ../tbtc-diagrams.tex

\tikz[
    group/.style={draw,rectangle,loosely dashed,inner sep=14pt,outer sep=0},
]{
    \node[leaf state] (start) {Start};
    \node[box state,right=of start] (awaiting setup) {Awaiting Signer Setup};
    \node[box state,above=of awaiting setup] (awaiting deposit proof) {Awaiting BTC Deposit Proof};

    \node[box state,above right=3cm of awaiting deposit proof] (active) {Active};

    \node[leaf state,left=of awaiting deposit proof] (failed setup) {Failed Setup};

    \node[box state,above=of awaiting deposit proof] (custodian margin called) {Custodian Margin Called};
    \node[box state,above=2cm of active] (fraud pre-liquidation) {Fraud pre-liquidation};

    \node[box state,left=of fraud pre-liquidation] (awaiting owner option) {Awaiting owner option};
    \node[box state,above left=of awaiting owner option] (liquidation auction) {Liquidation auction in-progress};
    \node[leaf state,above=3cm of awaiting owner option] (liquidated) {Liquidated};

    \node[box state,below right=of active] (owner margin called) {Owner Margin Called};
    \node[box state,below=of owner margin called] (change owner auction) {Change Owner Auction};

    \node[box state,above right=of fraud pre-liquidation] (awaiting signature) {Awaiting Signature};
    \node[box state,right=of awaiting signature] (awaiting redemption proof) {Awaiting Redemption Proof};
    \node[leaf state,below=of awaiting redemption proof] (redeemed) {Redeemed};

    \node[group,
          fit=(liquidation auction)
              (liquidated)
              (awaiting owner option)]
          (liquidation) {};
    \node[above right] at (liquidation.south west) {Liquidation};
    \node[group,
          dotted,
          fit=(custodian margin called)
              (liquidation auction)
              (liquidated)
              (awaiting owner option)]
          (undercollateralized) {};
    \node[above right] at (undercollateralized.south west) {Undercollateralized};
    \node[group,
          fit=(awaiting signature)
              (awaiting redemption proof)
              (redeemed)]
          (redemption) {};
    \node[above] at (redemption.south) {Redemption};
    \node[group,
          inner sep=22pt,
          fit=(owner margin called)
              (change owner auction)]
          (unmaintained) {};
    \node[above] at (unmaintained.south) {Unmaintained};


    \path [->] (start) edge [] (awaiting setup)

               (awaiting setup) edge [] (awaiting deposit proof)
               (awaiting setup) edge [] (failed setup)

               (awaiting deposit proof) edge [] (failed setup)
               (awaiting deposit proof) edge [bend right] (active)

               (active) edge [bend right=10] (custodian margin called)
               (active) edge (fraud pre-liquidation)
               (active) edge [bend right=45] (awaiting signature)
               (active) edge [bend left=15] (owner margin called)

               (custodian margin called) edge [bend right=10] (active)
               (custodian margin called) edge [bend right=10] (fraud pre-liquidation)
               (custodian margin called) edge (awaiting owner option)

               (awaiting owner option) edge [bend left] (liquidation auction)
               (awaiting owner option) edge (liquidated)

               (liquidation auction) edge [bend left=20] (liquidated)

               (owner margin called) edge [bend left=15] (active)
               (owner margin called) edge (change owner auction)
               % Curl this all the awy around the other nodes.
               (owner margin called) edge [-.,bend right=60] ($(redemption.north east)+(6pt,6pt)$)
                                                              ($(redemption.north east)+(6pt,6pt)$)
                                     edge [bend right=70] (fraud pre-liquidation)

               (change owner auction) edge [bend left=35] (active)
               % Curl this all the awy around the other nodes.
               (change owner auction) edge [-.,bend right=60] ($(redemption.north east)+(36pt,36pt)$)
                                                              ($(redemption.north east)+(36pt,36pt)$)
                                      edge [bend right=80] (fraud pre-liquidation)

               (awaiting signature) edge [bend right=10] (awaiting redemption proof)
               (awaiting signature) edge (redeemed)
               (awaiting signature) edge [bend right=35] (fraud pre-liquidation)

               (awaiting redemption proof) edge [bend right=10] (awaiting signature)
               (awaiting redemption proof) edge (redeemed)
               (awaiting redemption proof) edge [bend right=60] (fraud pre-liquidation)

               (fraud pre-liquidation) edge (awaiting owner option)
               ;
}

  \caption{\label{fig:deposit-state-machine}Deposit state machine.}
\end{sidewaysfigure}

\end{document}
